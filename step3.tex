\section{Третий шаг}

Сделаем подстановку в виде

$$
	\left\{
		\begin{aligned}
			s = \dfrac{s_0}{\tau} + \alpha \tau^{r-1} \\
			y = \dfrac{y_0}{\tau^2} + \beta \tau^{r-2} \\
			x = \dfrac{x_0}{\tau^2} + \theta \tau^{r-2} 
		\end{aligned}
	\right.
$$

Найдём $ \dot s $, $ \dot y $, $ \dot x $:

$$
	\left\{
		\begin{aligned}
			\dot s &= -\dfrac{s_0}{\tau^{2}} + \alpha (r-1) \tau^{r-2} \\
			\dot y &= -\dfrac{2y_0}{\tau^{3}} + \beta (r-2) \tau^{r-3} \\
			\dot x &= -\dfrac{2x_0}{\tau^{3}} + \theta (r-2) \tau^{r-3}
		\end{aligned}
	\right.
$$

Полученные выражения подставим в исходную систему (\ref{eq:original}):

\begin{equation*}
\label{eq:step3}
	\left\{
		\begin{aligned}
			-\dfrac{s_0}{\tau^{2}} + \alpha (r-1) \tau^{r-2} &= a \left( \dfrac{y_0}{\tau^2} + \beta \tau^{r-2} \right) + \gamma \\
			-\dfrac{2y_0}{\tau^{3}} + \beta (r-2) \tau^{r-3} &= -c \left( \dfrac{y_0}{\tau^2} + \beta \tau^{r-2} \right) - \left( \dfrac{s_0}{\tau} + \alpha \tau^{r-1} \right) - \left( \dfrac{x_0}{\tau^2} + \theta \tau^{r-2} \right) \left( \dfrac{s_0}{\tau} + \alpha \tau^{r-1} \right) \\
			-\dfrac{2x_0}{\tau^{3}} + \theta (r-2) \tau^{r-3} &= -c \left( \dfrac{x_0}{\tau^2} + \theta \tau^{r-2} \right) + \left( \dfrac{y_0}{\tau^2} + \beta \tau^{r-2} \right) \left( \dfrac{s_0}{\tau} + \alpha \tau^{r-1} \right)
		\end{aligned}
	\right.
\end{equation*}

Немного преобразуем данную систему

$$
	\left\{
		\begin{aligned}
			-\dfrac{s_0}{\tau^{2}} + \alpha (r-1) \tau^{r-2} &= a \left( \dfrac{y_0}{\tau^2} + \beta \tau^{r-2} \right) + \gamma \\
			-\dfrac{2y_0}{\tau^{3}} + \beta (r-2) \tau^{r-3} &= -c \left( \dfrac{y_0}{\tau^2} + \beta \tau^{r-2} \right) - \left( \dfrac{s_0}{\tau} + \alpha \tau^{r-1} \right) - \left( \dfrac{x_0s_0}{\tau^3} + \tau^{r-3}(x_0\alpha + s_0\theta) + \alpha\theta\tau^{2r-3} \right) \\
			-\dfrac{2x_0}{\tau^{3}} + \theta (r-2) \tau^{r-3} &= -c \left( \dfrac{x_0}{\tau^2} + \theta \tau^{r-2} \right) + \left( \dfrac{y_0s_0}{\tau^3} + \tau^{r-3}(y_0\alpha + s_0\beta) + \alpha\beta\tau^{2r-3} \right)
		\end{aligned}
	\right.
$$

Рассмотрим коэффициенты при $ \tau^{r-2} $, $ \tau^{r-3} $, $ \tau^{r-3} $ в первом, во втором и в третьем уравнении соответственно

\begin{equation*}
\label{eq:step2_coeff}
	\left\{
		\begin{aligned}
			\alpha (r-1) &= \alpha \beta \\
			\beta  (r-2) &= -\alpha x_0 - \theta s_0 \\
			\theta (r-2) &= \alpha y_0 + \beta s_0
		\end{aligned}
	\right.
\end{equation*}

\newpage

Перепишем данную систему в следующем виде:

\begin{equation*}
\label{eq:step2_coeff_determinant}
	\left\{
		\begin{aligned}
			\alpha (r-1) - a \beta + 0 &= 0 \\
			\alpha x_0 + \beta (r-2) + \theta s_0 &= 0 \\
			-\alpha y_0	- \beta s_0 + \theta (r-2) &= 0 
		\end{aligned}
	\right.
\end{equation*}

Составим матрицу из коэффициентов при $ \alpha $, $ \beta $, $ \theta $ 

\begin{equation}
\label{eq:determinant}
	\left(
		\begin{array}{ccc}
			r-1 & -a & 0 \\
			x_0 & r-2 & s_0 \\
			-y_0 & -s_0 & r-2 \\
		\end{array}
	\right)
\end{equation}

Найдем определитель данной матрицы и приравняем его к нулю:

$ (r-1)(r-2)(r-2) + as_0y_0 + s^2_0(r-1) + ax_0(r-2) = 0 $

$ (r-1)(r^2-4r+4) + as_0y_0 + rs^2_0 - s^2_0 + rax_0 - 2ax_0 = 0 $

$ r^3 - 4r^2 + 4r - r^2 + 4r - 4 + as_0y_0 + rs^2_0 - s^2_0 + rax_0 - 2ax_0 = 0 $

$ r^3 - 5r^2 + (8 + s^2_0 + ax_0)r - s^2_0 + as_0y_0 - 2ax_0 - 4 = 0 $

При подстановке в данное уравнение выражений для $ x_0 $, $ y_0 $, $ s_0 $, уравнение разбивается на 2.

\vspace{1mm}
\textit{Первый случай:} $ x_0 = -\frac{2}{a} $, $ y_0 = \frac{2}{a}i $, $ s_0 = -2i $:
\vspace{2mm}

$ r^3 - 5r^2 + (8 - 4 + a(-\frac{2}{a}))r - 4 + a(-2i)(\frac{2}{a}i) + 4 + 2a(\frac{2}{a}) = 0 $

$ r^3 - 5r^2 + 2r + 8 = 0 $

\vspace{1mm}
\textit{Второй случай:} $ x_0 = -\frac{2}{a} $, $ y_0 = -\frac{2}{a}i $, $ s_0 = 2i $:
\vspace{2mm}

$ r^3 - 5r^2 + (8 - 4 + a(-\frac{2}{a}))r - 4 + a(2i)(-\frac{2}{a}i) + 4 + 2a(\frac{2}{a}) = 0 $

$ r^3 - 5r^2 + 2r + 8 = 0 $

\vspace{1mm}

Видно, что уравнения получились одинаковые. Корни уравнения:

$$
	\left[\
		\begin{aligned}
			r &= - 1 \\
			r &= 2 \\
			r &= 4
		\end{aligned}
	\right.
$$