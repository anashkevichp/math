\section{Математика}

\begin{itemize}

\item{Вращающееся магнитное поле, схема обмоток.}

\item{Cистема уравнений из физического смысла:}
$$
        \left\{
                \begin{aligned}
                        L \dfrac{di_1(t)}{dt} + Ri_1(t) = e + SB(\sin\theta(t)) \dot \theta (t), \\
                        L \dfrac{di_2(t)}{dt} + Ri_2(t) = e + SB(\cos\theta(t)) \dot \theta (t),
\\
  						I \ddot \theta = -\beta SB(i_1(t)\sin\theta + i_2(t)\cos\theta) - M.
                \end{aligned}
        \right.
$$
Сделав замены в виде:
$ s = \dot \theta_1 = \dot{(-\theta)}, a = \dfrac{\beta(SB)^2}{IL}, c = \dfrac{R}{L}, $
$ x = \dfrac{L}{SB} (i_1\cos\theta_1 + i_2\sin\theta_1),$
$ y = \dfrac{L}{SB} (-i_1\sin\theta_1 + i_2\cos\theta_1) $, получим \textbf{исходную систему}.

\item{Получение исходной системы}


$$
        \left\{
                \begin{aligned}
                        \dot s &= ay + \gamma \\
                        \dot y &= -cy -s -xs \\
                        \dot x &= cx + ys
                \end{aligned}
        \right. 
$$

\item{Применение метода Пенлеви к системе}

  \begin{enumerate}
    
    \item{Первый шаг}

      Сделаем подстановку (на слайде), приведем члены уравнений к общим знаменателям, приравняем показатели степеней при $ \tau $.
      Учитывая, что $ k, l, m $ целые, получим: $ m = 2, k = 1, l = 2 $.

      Приравняем коэффициенты при одинаковых степенях $ \tau $, получим два случая (на слайде).

    \item{Второй шаг}

      Исходная система и подстановка.

      Cоставим матрицу коэффициентов. Она имеет решение -- её определитель должен быть равен нулю. Корни уравнения. Замечание про инвариантность (независимость) относительно результатов предыдущего шага.

    \item{Третий шаг}

      Описание подстановки. 

      Система коэффициентов.

      Анализ решения системы.
      
      \begin{itemize}
      \item
Степени $ \tau $, при которых следует искать произвольные постоянные, равны значениям r, полученным в шаге 2: 
\begin{center}
$ r = -1, 2, 4. $ 
\end{center}

\item
Действительно, среди полученных коэффициентов имеется одна произвольная постоянная --- $ y_0 $ при $ r = 2 $.

\item
Таким образом, мы имеем две произвольные постоянные: $ t_0 $ и $ y_0 $.
	\end{itemize}
      
  \end{enumerate}



\end{itemize}
