\section{Асинхронный двигатель}

\begin{itemize}
\item
Асинхронный двигатель --- электрическая машина переменного тока, частота вращения ротора которой не равна частоте вращения магнитного поля, создаваемого током обмотки статора.

\item
Основные части электродвигателя --- неподвижный статор и подвижный ротор.

ЭДС вызывают в короткозамкнутой обмотке ротора токи, которые, взаимодействуя с магнитным полем, являются причиной возникновения электрических сил и обусловливают появление момента вращения ротора. Этот момент вращения имеет то же самое направление, что и вращающееся поле, захватывает ротор асинхронного электродвигателя в этом направлении вращения и создает его ускорение.

\item
При вращении ротора -- в момент разбега с возрастающей частотой вращения -- относительная скорость по сравнению со скоростью вращающегося поля постепенно снижается. Однако ротор электродвигателя не может достичь синхронной частоты вращения поля, так как в этом случае в обмотке ротора на наводились бы ЭДС, не было бы тока и не создавался бы момент вращения.

В режиме холостого хода асинхронный электродвигатель имеет частоту вращения ротора $ n $, лишь незначительно отклоняющуюся от синхронной частоты вращения $ n_c $ из-за трения. 

Разница между относительными частотами вращения $ n_c - n $ называется \textbf{скольжением двигателя} s, т.е. $ s = 1 - \dfrac{n}{n_c} $ или в процентах $ s = 100 * (1- \dfrac{n}{n_c})\% $

\end{itemize}
