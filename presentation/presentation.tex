\documentclass[14pt]{beamer}
\usepackage[T2A]{fontenc}
\usepackage[utf8]{inputenc}
\usepackage[english,russian]{babel}
\usepackage{amssymb,amsfonts,amsmath,mathtext}
\usepackage{cite,enumerate,float,indentfirst}

\graphicspath{{images/}}

\usecolortheme{seagull}

\setbeamercolor{footline}{fg=gray}

\setbeamertemplate{footline}{
  \leavevmode%
  \hbox{%
  \begin{beamercolorbox}[wd=1\paperwidth,ht=2.25ex,dp=1ex,right]{}%
  Стр. \insertframenumber{} из \inserttotalframenumber \hspace*{2ex}
  \end{beamercolorbox}}%
  \vskip0pt%
}

\newcommand{\itemi}{\item[\checkmark]}

\author{\small{%
Тема нашей презентации\\
\vspace{50pt}
\emph{Выступающиe:}~П. С. Анашкевич, Р. И. Будный\\%
\emph{Руководитель:}~проф.,~к.ф.-м.н.~В.В.Цегельник}\\%
\vspace{30pt}%
}
\date{\small{Минск, 2013}}

\begin{document}

\maketitle

\begin{frame}
\frametitle{Постановка задачи}



\end{frame}

\begin{frame}
\frametitle{Асинхронный двигатель}

Устройство АД

\end{frame}

\begin{frame}
\frametitle{Асинхронный двигатель}

Определение

\end{frame}


\begin{frame}
\frametitle{Асинхронный двигатель}

Принцип работы

\end{frame}


\begin{frame}
\frametitle{Метод Пенлеви}

Фото + годы жизни

\begin{figure}[H]
  \center
  \includegraphics[width=0.8\linewidth]{latex}
\end{figure}

\end{frame}

\begin{frame}

\frametitle{Метод Пенлеви}

Сам метод

\end{frame}

\begin{frame}

\frametitle{Метод Пенлеви}

Зачем он нам?

\end{frame}


\begin{frame}
\frametitle{Схема АД}

\end{frame}


\begin{frame}

\frametitle{Физическое описание}

Система уравнений из физического смысла

\end{frame}


\begin{frame}

\frametitle{Исходная система}

Сама система

\end{frame}



\begin{frame}

\frametitle{Метод Пенлеви: шаг 1}

Исходная система + подстановка

\end{frame}



\begin{frame}

\frametitle{Метод Пенлеви: шаг 2}

Результат предыдущего шага, приравняем коэффициенты при одинаковых степенях $ \tau $ -- система.

\end{frame}

\begin{frame}

\frametitle{Метод Пенлеви: шаг 2}

Результаты шага 2.

\end{frame}



\begin{frame}

\frametitle{Метод Пенлеви: шаг 3}

Исходная система + подстановка.

\end{frame}



\begin{frame}

\frametitle{Метод Пенлеви: шаг 3}

Определитель должен быть равен нулю + характеристическое уравнение.

\end{frame}



\begin{frame}

\frametitle{Метод Пенлеви: шаг 3}

Корни уравнения.

\end{frame}



\begin{frame}

\frametitle{Метод Пенлеви: шаг 4}

Описание подстановки.

\end{frame}


\begin{frame}

\frametitle{Метод Пенлеви: шаг 4}

Полученные коэффициенты.

\end{frame}


\begin{frame}

\frametitle{Анализ решения}

Анализ решения системы.

\end{frame}

\begin{frame}

\frametitle{Выводы}

Выводы

\end{frame}


\begin{frame}
\begin{center}
Спасибо за внимание!
\end{center}
\end{frame}

\end{document} 