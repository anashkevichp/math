\section{Шаг первый}

Исходная система уравнений:

\begin{equation}
\label{eq:original}
	\left\{
		\begin{aligned}
			\dot s &= ay + \gamma \\
			\dot y &= -cy -s -xs \\
			\dot x &= cx + ys
		\end{aligned}
	\right.
\end{equation}

Сделаем подстановку в виде $ s = \dfrac{s_0}{\tau^k} $, $ y = \dfrac{y_0}{\tau^l} $, $ x = \dfrac{x_0}{\tau^m} $

Найдём $ \dot s $, $ \dot y $, $ \dot x $:

$$
	\left\{
		\begin{aligned}
			\dot s &= -\dfrac{ks_0}{\tau^{k+1}} \\
			\dot y &= -\dfrac{ly_0}{\tau^{l+1}} \\
			\dot x &= -\dfrac{mx_0}{\tau^{m+1}} 
		\end{aligned}
	\right.
$$

Полученные выражения подставим в исходную систему (\ref{eq:original}):

\begin{equation}
\label{eq:step1}
	\left\{
		\begin{aligned}
			-\dfrac{ks_0}{\tau^{k+1}} &= a \dfrac{y_0}{\tau^l} + \gamma \\
			-\dfrac{ly_0}{\tau^{l+1}} &= -c \dfrac{y_0}{\tau^l} - \dfrac{s_0}{\tau^k} - \dfrac{x_0}{\tau^m} \dfrac{s_0}{\tau^k} \\
			-\dfrac{mx_0}{\tau^{m+1}} &= -c \dfrac{x_0}{\tau^m} + \dfrac{y_0}{\tau^l} \dfrac{s_0}{\tau^k}
		\end{aligned}
	\right.
\end{equation}

Учитывая, что $ k $, $ l $, $ m \in \textbf{Z} $, получим:
 
$$
	\left\{
		\begin{aligned}
			k + 1 &= l \\
			l + 1 &= k \text{  или  } l + 1 = m + k \\
			m + 1 &= l + k 
		\end{aligned}
	\right.
$$

Отбросив $ l + 1 = k $, и решив данную систему, однозначно получим что $ m = 2 $, $ k = 1 $, $ l = 2 $.

В итоге получим систему:

\begin{equation}
\label{eq:step1_results}
	\left\{
		\begin{aligned}
			-\dfrac{s_0}{\tau^2} &= a \dfrac{y_0}{\tau^2} + \gamma \\
			-\dfrac{2y_0}{\tau^3} &= -c \dfrac{y_0}{\tau^2} - \dfrac{s_0}{\tau} - \dfrac{x_0 s_0}{\tau^3} \\
			-\dfrac{2x_0}{\tau^{3}} &= -c \dfrac{x_0}{\tau^2} + \dfrac{y_0 s_0}{\tau^3}
		\end{aligned}
	\right.
\end{equation}