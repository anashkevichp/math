\section{Третий шаг}

Подставим в систему~\ref{eq:original} следующие выражения:

\begin{equation*}
	\left\{
		\begin{aligned}
			s = \dfrac{s_0}{\tau} + \alpha \tau^{r-1} \\
			y = \dfrac{y_0}{\tau^2} + \beta \tau^{r-2} \\
			x = \dfrac{x_0}{\tau^2} + \theta \tau^{r-2} 
		\end{aligned}
	\right.
\end{equation*}

Найдём $ \dot s, \dot y, \dot x $:
\begin{equation*}
	\left\{
		\begin{aligned}
			\dot s &= -\dfrac{s_0}{\tau^{2}} + \alpha (r-1) \tau^{r-2} \\
			\dot y &= -\dfrac{2y_0}{\tau^{3}} + \beta (r-2) \tau^{r-3} \\
			\dot x &= -\dfrac{2x_0}{\tau^{3}} + \theta (r-2) \tau^{r-3}
		\end{aligned}
	\right.
\end{equation*}

Полученные выражения подставим в исходную систему~\ref{eq:original}:
\begin{equation}
  \label{eq:step3}
  \left\{
    \begin{aligned}
      -\dfrac{s_0}{\tau^{2}} + \alpha (r-1) \tau^{r-2} &= a \left( \dfrac{y_0}{\tau^2} + \beta \tau^{r-2} \right) + \gamma \\
      -\dfrac{2y_0}{\tau^{3}} + \beta (r-2) \tau^{r-3} &= -c \left( \dfrac{y_0}{\tau^2} + \beta \tau^{r-2} \right) - \left( \dfrac{s_0}{\tau} + \alpha \tau^{r-1} \right) - \left( \dfrac{x_0}{\tau^2} + \theta \tau^{r-2} \right) \left( \dfrac{s_0}{\tau} + \alpha \tau^{r-1} \right) \\
      -\dfrac{2x_0}{\tau^{3}} + \theta (r-2) \tau^{r-3} &= -c \left( \dfrac{x_0}{\tau^2} + \theta \tau^{r-2} \right) + \left( \dfrac{y_0}{\tau^2} + \beta \tau^{r-2} \right) \left( \dfrac{s_0}{\tau} + \alpha \tau^{r-1} \right)
    \end{aligned}
  \right.
\end{equation}

Раскроем скобки:
\begin{equation*}
  \left\{
    \begin{aligned}
      -\dfrac{s_0}{\tau^{2}} + \alpha (r-1) \tau^{r-2} &= a \left( \dfrac{y_0}{\tau^2} + \beta \tau^{r-2} \right) + \gamma \\
      -\dfrac{2y_0}{\tau^{3}} + \beta (r-2) \tau^{r-3} &= -c \left( \dfrac{y_0}{\tau^2} + \beta \tau^{r-2} \right) - \left( \dfrac{s_0}{\tau} + \alpha \tau^{r-1} \right) - \left( \dfrac{x_0s_0}{\tau^3} + \tau^{r-3}(x_0\alpha + s_0\theta) + \alpha\theta\tau^{2r-3} \right) \\
      -\dfrac{2x_0}{\tau^{3}} + \theta (r-2) \tau^{r-3} &= -c \left( \dfrac{x_0}{\tau^2} + \theta \tau^{r-2} \right) + \left( \dfrac{y_0s_0}{\tau^3} + \tau^{r-3}(y_0\alpha + s_0\beta) + \alpha\beta\tau^{2r-3} \right)
    \end{aligned}
  \right.
\end{equation*}

Приравняем коэффициенты при $ \tau^{r-2}, \tau^{r-3}, \tau^{r-3} $ в каждом уравнении системы~\ref{eq:step3}:
\begin{equation*}
\label{eq:step2_coeff}
	\left\{
		\begin{aligned}
			\alpha (r-1) &= a \beta \\
			\beta  (r-2) &= -\alpha x_0 - \theta s_0 \\
			\theta (r-2) &= \alpha y_0 + \beta s_0
		\end{aligned}
	\right.
\end{equation*}

Перепишем данную систему в следующем виде:
\begin{equation*}
\label{eq:step2_coeff_determinant}
	\left\{
		\begin{aligned}
			\alpha (r-1) - a \beta + 0 &= 0 \\
			\alpha x_0 + \beta (r-2) + \theta s_0 &= 0 \\
			-\alpha y_0	- \beta s_0 + \theta (r-2) &= 0 
		\end{aligned}
	\right.
\end{equation*}

\newpage

Составим матрицу из коэффициентов при $ \alpha $, $ \beta $, $ \theta $:
\begin{equation*}
\label{eq:determinant}
	\left(
		\begin{array}{ccc}
			r-1 & -a & 0 \\
			x_0 & r-2 & s_0 \\
			-y_0 & -s_0 & r-2 \\
		\end{array}
	\right)
\end{equation*}

Найдем определитель данной матрицы и приравняем его к нулю:

\begin{align}
\nonumber
(r-1)(r-2)(r-2) + as_0y_0 + s^2_0(r-1) + ax_0(r-2) = 0 \\ 
\nonumber
(r-1)(r^2-4r+4) + as_0y_0 + rs^2_0 - s^2_0 + rax_0 - 2ax_0 = 0 \\
\nonumber
r^3 - 4r^2 + 4r - r^2 + 4r - 4 + as_0y_0 + rs^2_0 - s^2_0 + rax_0 - 2ax_0 = 0 \\ 
\label{eq:char_polinom}
r^3 - 5r^2 + (8 + s^2_0 + ax_0)r - s^2_0 + as_0y_0 - 2ax_0 - 4 = 0 
\end{align}

При подстановке в уравнение~\ref{eq:char_polinom} выражений~\ref{eq:step2_a},~\ref{eq:step2_b} для $ x_0 $, $ y_0 $, $ s_0 $ получаем два случая:
\begin{itemize}

\item при $ x_0 = -\frac{2}{a} $, $ y_0 = \frac{2}{a}i $, $ s_0 = -2i $:
  \begin{align}
    \nonumber
    & r^3 - 5r^2 + (8 - 4 + a(-\frac{2}{a}))r - 4 + a(-2i)(\frac{2}{a}i) + 4 + 2a(\frac{2}{a}) = 0 \\
    \label{eq:char_polinom_a}
    & r^3 - 5r^2 + 2r + 8 = 0 
  \end{align}

\item при $ x_0 = -\frac{2}{a} $, $ y_0 = -\frac{2}{a}i $, $ s_0 = 2i $:
  \begin{align}
    \nonumber
    & r^3 - 5r^2 + (8 - 4 + a(-\frac{2}{a}))r - 4 + a(2i)(-\frac{2}{a}i) + 4 + 2a(\frac{2}{a}) = 0 \\
    \label{eq:char_polinom_b}
    & r^3 - 5r^2 + 2r + 8 = 0 
  \end{align}

\end{itemize}

Видно, что уравнения~\ref{eq:char_polinom_a} и~\ref{eq:char_polinom_b} имеют одинаковый вид.
Определим корни этих уравнений:

\begin{equation}
  \left[\
    \begin{aligned}
      r &= - 1 \\
      r &= 2 \\
      r &= 4
    \end{aligned}
  \right.
\end{equation}

\pagebreak