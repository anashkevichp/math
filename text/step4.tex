\section{Четвертый шаг}

Сделаем подстановку в виде бесконечных рядов для $ s $, $ y $, $ x $:

\begin{equation*}
  \left\{
    \begin{aligned}
      s &= \dfrac{s_{-1}}{\tau} + s_0 + s_1\tau + s_2\tau^2 + s_3\tau^3 + \ldots \\
      y &= \dfrac{y_{-2}}{\tau^2} + \dfrac{y_{-1}}{\tau} + y_0 + y_1\tau + y_2\tau^2 + y_3\tau^3 + \ldots \\
      x &= \dfrac{x_{-2}}{\tau^2} + \dfrac{x_{-1}}{\tau} + x_0 + x_1\tau + x_2\tau^2 + x_3\tau^3 + \ldots
    \end{aligned}
  \right.
\end{equation*}

Найдём $ \dot s $, $ \dot y $, $ \dot x $:
\begin{equation*}
	\left\{
		\begin{aligned}
			\dot s &= -\dfrac{s_{-1}}{\tau^{2}} + s_1 + 2s_2\tau + 3s_3\tau^2 + \ldots \\
			\dot y &= -\dfrac{2y_{-2}}{\tau^{3}} - \dfrac{y_{-1}}{\tau^2} + y_1 + 2y_2\tau + 3y_3\tau^2 + \ldots \\
			\dot x &= -\dfrac{2x_{-2}}{\tau^{3}} - \dfrac{x_{-1}}{\tau^2} + x_1 + 2x_2\tau + 3x_3\tau^2 + \ldots
		\end{aligned}
	\right.
\end{equation*}

Полученные выражения подставим в исходную систему (\ref{eq:original}):
\begin{equation*}
  \label{eq:step4}
  \left\{
    \begin{aligned}
      -\dfrac{s_{-1}}{\tau^{2}} + s_1 + 2s_2\tau + 3s_3\tau^2 + \ldots = a \left( \dfrac{y_{-2}}{\tau^2} + \dfrac{y_{-1}}{\tau} + y_0 + y_1\tau + y_2\tau^2 + y_3\tau^3 + \ldots \right) + \gamma \\			
      -\dfrac{2y_{-2}}{\tau^{3}} - \dfrac{y_{-1}}{\tau^2} + y_1 + 2y_2\tau + 3y_3\tau^2 + \ldots = -c \left( \dfrac{y_{-2}}{\tau^2} + \dfrac{y_{-1}}{\tau} + y_0 + y_1\tau + y_2\tau^2 + y_3\tau^3 + \ldots \right) - \\ - \left( \dfrac{s_{-1}}{\tau} + s_0 + s_1\tau + s_2\tau^2 + s_3\tau^3 + \ldots \right) - \left( \dfrac{x_{-2}}{\tau^2} + \dfrac{x_{-1}}{\tau} + x_0 + x_1\tau + x_2\tau^2 + x_3\tau^3 + \ldots \right) \cdot \\ \cdot \left( \dfrac{s_{-1}}{\tau} + s_0 + s_1\tau + s_2\tau^2 + s_3\tau^3 + \ldots \right) \\	
      -\dfrac{2x_{-2}}{\tau^{3}} - \dfrac{x_{-1}}{\tau^2} + x_1 + 2x_2\tau + 3x_3\tau^2 + \ldots = -c \left( \dfrac{x_{-2}}{\tau^2} + \dfrac{x_{-1}}{\tau} + x_0 + x_1\tau + x_2\tau^2 + x_3\tau^3 + \ldots \right) + \\ + \left( \dfrac{y_{-2}}{\tau^2} + \dfrac{y_{-1}}{\tau} + y_0 + y_1\tau + y_2\tau^2 + y_3\tau^3 + \ldots \right) \cdot \left( \dfrac{s_{-1}}{\tau} + s_0 + s_1\tau + s_2\tau^2 + s_3\tau^3 + \ldots \right)
    \end{aligned}
  \right.
\end{equation*}

\pagebreak 

Рассмотрим коэффициенты при одинаковых степенях $ \tau $, начиная с наинизшей.
Составим соответствующие системы уравнений:
\begin{equation*}
	\begin{aligned}
		(r = 0) &:		
		\left\{
			\begin{aligned}
				-s_{-1} &= ay_{-2} \\
				-2y_{-2} &= x_{-2} s_{-1} \\
				-2x_{-2} &= y_{-2} s_{-1}
			\end{aligned}
		\right.
		\\		
		(r = 1) &:	
		\left\{
			\begin{aligned}
				0 &= ay_{-1} \\
				-y_{-1} &= -cy_{-2} - x_{-2}s_0 - x_{-1}s_{-1}  \\
				-x_{-1} &= -cx_{-2} + y_{-2}s_0 + y_{-1}s_{-1}
			\end{aligned}
		\right.
		\\
		(r = 2) &:	
		\left\{
			\begin{aligned}
				s_1 &= ay_{0} + \gamma \\
				0 &= -cy_{-1} - s_{-1} - x_{-2}s_1 - x_{-1}s_0 - x_0s_{-1}  \\
				0 &= -cx_{-1} + y_{-2}s_1 + y_{-1}s_0 + y_{0}s_{-1}
			\end{aligned}
		\right.
		\\
		(r = 3) &:	
		\left\{
			\begin{aligned}
				2s_2 &= ay_{1} \\
				y_1 &= -cy_{0} - s_0 - x_{-2}s_2 - x_{-1}s_1 - x_0s_0 - x_1s_{-1}  \\
				x_1 &= -cx_{0} + y_{-2}s_2 + y_{-1}s_1 + y_{0}s_0 + y_1s_{-1}
			\end{aligned}
		\right.
		\\
		(r = 4) &:	
		\left\{
			\begin{aligned}
				3s_3 &= ay_{2} \\
				2y_2 &= -cy_{1} - s_1 - x_{-2}s_3 - x_{-1}s_2 - x_0s_1 - x_1s_0 - x_2s_{-1}  \\
				2x_2 &= -cx_{1} + y_{-2}s_3 + y_{-1}s_2 + y_{0}s_1 + y_1s_0 + y_2s_{-1}
			\end{aligned}
		\right.
		\\
		(r = 5) &:	
		\left\{
			\begin{aligned}
				4s_4 &= ay_{3} \\
				3y_3 &= -cy_{2} - s_2 - x_{-2}s_4 - x_{-1}s_3 - x_0s_2 - x_1s_1 - x_2s_0 - x_3s_{-1} \\
				3x_3 &= -cx_{2} + y_{-2}s_4 + y_{-1}s_3 + y_{0}s_2 + y_1s_1 + y_2s_0 + y_3s_{-1}
			\end{aligned}
		\right.
		\\
		(r = 6) &:	
		\left\{
			\begin{aligned}
				5s_5 &= ay_{4} \\
				4y_4 &= -cy_{3} - s_3 - x_{-2}s_5 - x_{-1}s_4 - x_0s_3 - x_1s_2 - x_2s_1 - x_3s_0 - x_4s_{-1} \\
				4x_4 &= -cx_{3} + y_{-2}s_5 + y_{-1}s_4 + y_{0}s_3 + y_1s_2 + y_2s_1 + y_3s_0 + y_4s_{-1}
			\end{aligned}
		\right. 
	\end{aligned}
\end{equation*}

\begin{center}
	$ \ldots $
\end{center}
						
Поочередно рассмотрим каждую из систем, постепенно подставляя найденные значения.

Из первой $ (r = 0) $ системы получаем:
\vspace{0.5em}

\begin{minipage}[h!]{0.35\linewidth}
  \begin{equation}
    \label{eq:step4_a}
    \left\{
      \begin{aligned}
        x_{-2} &= \dfrac{2}{a} \\
        y_{-2} &= \dfrac{2}{a} \\
        s_{-1} &= -2 \\
      \end{aligned}
    \right.
  \end{equation}
\end{minipage}
\hfill
либо
\hfill
\begin{minipage}[h!]{0.35\linewidth}
  \begin{equation}
    \label{eq:step4_b}
    \left\{
      \begin{aligned}
        x_{-2} &= \dfrac{2}{a} \\
        y_{-2} &= -\dfrac{2}{a} \\
        s_{-1} &= 2 \\
      \end{aligned}
    \right.
  \end{equation}
\end{minipage}

\vspace{0.5em}

Для нахождения остальных коэффициентов ($ x_{-1}, y_{-1}, s_{0},
x_{0}, y_{0}, s_{1} $ и т.д.) необходимо подставить полученные на
данном шаге значения ($ x_{-2}, y_{-2}, s_{-1} $) в соответствующие
системы уравнений.

Рассмотрим два решения в отдельности.

\clearpage

Начнём с \textit{\textbf{первой ветки решения}}~\ref{eq:step4_a}. Дальнейшие подстановки дают нам следующие значения:
\begin{itemize}
\item Вторая система $ (r = 1) $:
\begin{equation*}
\left\{
	\begin{aligned}
		s_{0} &= \dfrac{c}{3} \\
		y_{-1} &= 0 \\
		x_{-1} &= \dfrac{4c}{3a}
	\end{aligned}
\right.
\end{equation*}

\item Найденные значения подставим в третью систему $ (r = 2) $.
  После преобразований одного из уравнений получаем:
\begin{equation*} 
	y_0 = y_0
\end{equation*} 

Таким образом, $ y_0 $ - произвольный постоянный коэффициент, а $ x_0, s_1 $ выражаются через $ y_0 $:
\begin{equation*}
\left\{
	\begin{aligned}
		y_{0} &- \text{произвольный} \\
		s_{1} &= ay_0 + \dfrac{2c^2}{3} \\
		x_{0} &= \dfrac{8c^2}{9a} - 1 + y_0
	\end{aligned}
\right.
\end{equation*}

\item Четвертая система $ (r = 3) $:
\begin{equation*}
\left\{
	\begin{aligned}
		y_1 &= -\dfrac{20c^3}{27} - cy_0 + \dfrac{c}{2} \\
		x_1 &= -\dfrac{4c^3}{27a} + \dfrac{cy_0}{3} + \dfrac{c}{2} \\
		s_2 &= -\dfrac{10c^3}{27} - \dfrac{cay_0}{2} + \dfrac{ca}{4}
	\end{aligned}
\right.
\end{equation*}

\item Подставив в пятую систему $ (r = 4) $ найденные выражения получим, что $ y_2,s_3 $ не зависит от $ y_0 $:
\begin{equation*}
\left\{
	\begin{aligned}
		x_2 &= -\dfrac{50c^4}{243} + \dfrac{14c^4}{243a} + \dfrac{c^2}{18} + \dfrac{ay^2_0}{2}\\
		y_2 &= \dfrac{10c^4}{81} + \dfrac{2c^4}{81a} - \dfrac{c^2}{3} \\
		s_3 &= \dfrac{10ac^4}{243} + \dfrac{2c^4}{243} - \dfrac{ac^2}{9}
	\end{aligned}
\right.
\end{equation*}

\item Шестая система $ (r = 5) $:
\begin{equation*}
\left\{
	\begin{aligned}
		x_3 &= f_1(a,c,y_0) \\
		y_3 &= f_2(a,c,y_0) \\
		s_4 &= f_3(a,c,y_0)
	\end{aligned}
\right.
\end{equation*}

\begin{center}
	$ \ldots $
\end{center}

\end{itemize}
\newpage

Рассмотрим \textit{\textbf{вторую ветку решения}}. Из первой системы ($ r = 0 $) получили систему (\ref{eq:step4_b}).
Дальнейшие подстановки дают нам следующие значения:
\begin{itemize}

\item Вторая система $ (r = 1) $:
\begin{equation*}
\left\{
	\begin{aligned}
		s_{0} &= -\dfrac{c}{3} \\
		y_{-1} &= 0 \\
		x_{-1} &= \dfrac{4c}{3a}
	\end{aligned}
\right.
\end{equation*}

\item Найденные значения подставим в третью систему $ (r = 2) $. После преобразований одного из уравнений получаем:
$$
	y_0 = y_0
$$ 

Таким образом $ y_0 $ --- произвольный постоянный коэффициент, а $ x_0, s_1 $ выражаются через $ y_0 $:
\begin{equation*}
\left\{
	\begin{aligned}
		y_{0} &- \text{произвольный} \\
		s_{1} &= ay_0 - \dfrac{2c^2}{3} \\
		x_{0} &= \dfrac{8c^2}{9a} - 1 - y_0
	\end{aligned}
\right.
\end{equation*}

\item Четвертая система $ (r = 3) $:
\begin{equation*}
\left\{
	\begin{aligned}
		y_1 &= \dfrac{20c^3}{27} - cy_0 - \dfrac{c}{2} \\
		x_1 &= -\dfrac{4c^3}{27a} - \dfrac{cy_0}{3} + \dfrac{c}{2} \\
		s_2 &= \dfrac{10c^3}{27} - \dfrac{cay_0}{2} - \dfrac{ca}{4}
	\end{aligned}
\right.
\end{equation*}

\item Подставив в пятую систему $ (r = 4) $ найденные выражения, получим, что $ y_2,s_3 $ не зависит от $ y_0 $:
\begin{equation*}
\left\{
	\begin{aligned}
		x_2 &= -\dfrac{50c^4}{243} + \dfrac{14c^4}{243a} + \dfrac{c^2}{18} + \dfrac{ay^2_0}{2}\\
		y_2 &= -\dfrac{10c^4}{81} - \dfrac{2c^4}{81a} + \dfrac{c^2}{3} \\
		s_3 &= -\dfrac{10ac^4}{243} - \dfrac{2c^4}{243} + \dfrac{ac^2}{9}
	\end{aligned}
\right.
\end{equation*}

\item Шестая система $ (r = 5) $:
\begin{equation*}
\left\{
	\begin{aligned}
		x_3 &= f_1(a,c,y_0) \\
		y_3 &= f_2(a,c,y_0) \\
		s_4 &= f_3(a,c,y_0)
	\end{aligned}
\right.
\end{equation*}

\begin{center}
	$ \ldots $
\end{center}

\end{itemize}