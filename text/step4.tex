\section{Четвертый шаг}

Сделаем подстановку в виде бесконечных рядов для $ s $, $ y $, $ x $ 

$$
	\left\{
		\begin{aligned}
			s &= \dfrac{s_{-1}}{\tau} + s_0 + s_1\tau + s_2\tau^2 + s_3\tau^3 + \ldots \\
			y &= \dfrac{y_{-2}}{\tau^2} + \dfrac{y_{-1}}{\tau} + y_0 + y_1\tau + y_2\tau^2 + y_3\tau^3 + \ldots \\
			x &= \dfrac{x_{-2}}{\tau^2} + \dfrac{x_{-1}}{\tau} + x_0 + x_1\tau + x_2\tau^2 + x_3\tau^3 + \ldots
		\end{aligned}
	\right.
$$

Найдём $ \dot s $, $ \dot y $, $ \dot x $:

$$
	\left\{
		\begin{aligned}
			\dot s &= -\dfrac{s_{-1}}{\tau^{2}} + s_1 + 2s_2\tau + 3s_3\tau^2 + \ldots \\
			\dot y &= -\dfrac{2y_{-2}}{\tau^{3}} - \dfrac{y_{-1}}{\tau^2} + y_1 + 2y_2\tau + 3y_3\tau^2 + \ldots \\
			\dot x &= -\dfrac{2x_{-2}}{\tau^{3}} - \dfrac{x_{-1}}{\tau^2} + x_1 + 2x_2\tau + 3x_3\tau^2 + \ldots
		\end{aligned}
	\right.
$$

\newpage

Полученные выражения подставим в исходную систему (\ref{eq:original}):

\begin{eqnarray*}
	\left\{
		\begin{aligned}
			-\dfrac{s_{-1}}{\tau^{2}} + s_1 + 2s_2\tau + 3s_3\tau^2 + \ldots = a \left( \dfrac{y_{-2}}{\tau^2} + \dfrac{y_{-1}}{\tau} + y_0 + y_1\tau + y_2\tau^2 + y_3\tau^3 + \ldots \right) + \gamma \\			
			-\dfrac{2y_{-2}}{\tau^{3}} - \dfrac{y_{-1}}{\tau^2} + y_1 + 2y_2\tau + 3y_3\tau^2 + \ldots = -c \left( \dfrac{y_{-2}}{\tau^2} + \dfrac{y_{-1}}{\tau} + y_0 + y_1\tau + y_2\tau^2 + y_3\tau^3 + \ldots \right) - \\ - \left( \dfrac{s_{-1}}{\tau} + s_0 + s_1\tau + s_2\tau^2 + s_3\tau^3 + \ldots \right) - \left( \dfrac{x_{-2}}{\tau^2} + \dfrac{x_{-1}}{\tau} + x_0 + x_1\tau + x_2\tau^2 + x_3\tau^3 + \ldots \right) \cdot \\ \cdot \left( \dfrac{s_{-1}}{\tau} + s_0 + s_1\tau + s_2\tau^2 + s_3\tau^3 + \ldots \right) \\	
			-\dfrac{2x_{-2}}{\tau^{3}} - \dfrac{x_{-1}}{\tau^2} + x_1 + 2x_2\tau + 3x_3\tau^2 + \ldots = -c \left( \dfrac{x_{-2}}{\tau^2} + \dfrac{x_{-1}}{\tau} + x_0 + x_1\tau + x_2\tau^2 + x_3\tau^3 + \ldots \right) + \\ + \left( \dfrac{y_{-2}}{\tau^2} + \dfrac{y_{-1}}{\tau} + y_0 + y_1\tau + y_2\tau^2 + y_3\tau^3 + \ldots \right) \cdot \left( \dfrac{s_{-1}}{\tau} + s_0 + s_1\tau + s_2\tau^2 + s_3\tau^3 + \ldots \right)
		\end{aligned}
	\right.
\end{eqnarray*}

Рассмотрим коэффициенты при одинаковых степенях $ \tau $.
 
$$
	\begin{aligned}
		\dfrac{1}{\tau^3} &:		
		\left\{
			\begin{aligned}
				0 &= a \cdot 0 \\
				-2y_{-2} &= x_{-2} s_{-1} \\
				-2x_{-2} &= y_{-2} s_{-1}
			\end{aligned}
		\right.
		\\		
		\dfrac{1}{\tau^2} &:	
		\left\{
			\begin{aligned}
				-s_{-1} &= ay_{-2} \\
				-y_{-1} &= -cy_{-2} - x_{-2}s_0 - x_{-1}s_{-1}  \\
				-x_{-1} &= -cx_{-2} + y_{-2}s_0 + y_{-1}s_{-1}
			\end{aligned}
		\right.
		\\
		\dfrac{1}{\tau} &:	
		\left\{
			\begin{aligned}
				0 &= ay_{-1} \\
				0 &= -cy_{-1} - s_{-1} - x_{-2}s_1 - x_{-1}s_0 - x_0s_{-1}  \\
				0 &= -cx_{-1} + y_{-2}s_1 + y_{-1}s_0 + y_{0}s_{-1}
			\end{aligned}
		\right.
		\\
		\tau^0 &:	
		\left\{
			\begin{aligned}
				s_1 &= ay_{0} + \gamma \\
				y_1 &= -cy_{0} - s_0 - x_{-2}s_2 - x_{-1}s_1 - x_0s_0 - x_1s_{-1}  \\
				x_1 &= -cx_{0} + y_{-2}s_2 + y_{-1}s_1 + y_{0}s_0 + y_1s_{-1}
			\end{aligned}
		\right.
		\\
		\tau &:	
		\left\{
			\begin{aligned}
				2s_2 &= ay_{1} \\
				2y_2 &= -cy_{1} - s_1 - x_{-2}s_3 - x_{-1}s_2 - x_0s_1 - x_1s_0 - x_2s_{-1}  \\
				2x_2 &= -cx_{1} + y_{-2}s_3 + y_{-1}s_2 + y_{0}s_1 + y_1s_0 + y_2s_{-1}
			\end{aligned}
		\right.
		\\
		\tau^2 &:	
		\left\{
			\begin{aligned}
				3s_3 &= ay_{2} \\
				3y_3 &= -cy_{2} - s_2 - x_{-2}s_4 - x_{-1}s_3 - x_0s_2 - x_1s_1 - x_2s_0 - x_3s_{-1} \\
				3x_3 &= -cx_{2} + y_{-2}s_4 + y_{-1}s_3 + y_{0}s_2 + y_1s_1 + y_2s_0 + y_3s_{-1}
			\end{aligned}
		\right.
		\\
		\tau^3 &:	
		\left\{
			\begin{aligned}
				4s_4 &= ay_{3} \\
				4y_4 &= -cy_{3} - s_3 - x_{-2}s_5 - x_{-1}s_4 - x_0s_3 - x_1s_2 - x_2s_1 - x_3s_0 - x_4s_{-1} \\
				4x_4 &= -cx_{3} + y_{-2}s_5 + y_{-1}s_4 + y_{0}s_3 + y_1s_2 + y_2s_1 + y_3s_0 + y_4s_{-1}
			\end{aligned}
		\right. 
	\end{aligned}
$$
\begin{center}
	$ \ldots $
\end{center}
						





